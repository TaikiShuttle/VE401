\documentclass[12pt]{report}
\usepackage[utf8]{inputenc}
\usepackage{graphicx}
\usepackage{amsmath}
\usepackage{longtable}
\usepackage{supertabular}
\usepackage{setspace}
\usepackage{geometry}
\usepackage{ulem}
\usepackage{lastpage}
\usepackage{fancyhdr}
\usepackage{indentfirst}
\usepackage{enumerate}
\usepackage{makecell}
\usepackage{floatrow}
\usepackage{caption}
\usepackage{multirow}
\usepackage{subfigure}
\usepackage{amssymb}
\usepackage{biblatex}
\graphicspath{images/}
\renewcommand\thesection{\arabic{section}}
\addbibresource{cite.bib}

\title
{
    {\includegraphics[scale = 0.5]{umji.png}\\}
    {\huge VE401 Exercise 1}
}
\author
{
        {\textbf{Group 38}} \\
        {Name: Zhou Hanquan}\\
        {ID:519370910059}\\
        {Name: Bian Hongyi}\\
        {ID:519370910066}\\
        {Name: Xu Ruiqi}\\
        {ID:520370910004}
        
}
\date{Feburary 2022}

\begin{document}

\maketitle
\newpage

\section*{Exercise 1.1}
We can calculate the posssibility using the Cardano's Principle.
$$
    P[A] = \frac{6 \times \binom{7}{6} \times 6!}{6^{7}*2} = 0.054
$$
Note that the 2 in the denominator comes from the fact that choosing any number in the remaining position leads to duplicates.

\section*{Exercise 1.2}
Use the independence of the probability, we can decompose the equation as 
$$
\begin{cases}
    P[\{1,3,5\}] = P[\{1\}] + P[\{3\}] + P[\{5\}] = \alpha\\
    P[\{1,2,3\}] = P[\{1\}] + P[\{2\}] + P[\{3\}] = \alpha\\
    P[\{2,4,5\}] = P[\{2\}] + P[\{4\}] + P[\{5\}] = \alpha
\end{cases}
$$
Together with the condition that all the posssibilities sum to 1, we have 4 equations and 5 unknowns.
So we need further condition. We assume the probability $P[\{2\}] = P[\{5\}] = \beta$
Then we can express all the probabilities as follows:
$$
\begin{cases}
    P[\{2\}] = P[\{5\}] = \beta \\
    P[\{3\}] = \alpha - \beta - P[\{1\}] \\
    P[\{4\}] = \alpha - 2\beta \\
    P[\{6\}] = 1 - P[\{1\}] - \alpha
\end{cases}
$$

\section*{Exercise 1.3}
We first carefully analyze the composition of the new results. We assume that one die gives result $x$ and the other die gives result $y$. Then
given the two dies rolling result $z$, we can write the probability as
$$
    P[\{z\}] = \sum_{x,y, \ x+y = z - 1 \ \text{mod} \ 6} P[\{x\}]P[\{y\}] 
$$
Further expand
$$
    P[\{x\}]P[\{y\}] = \frac{1}{36} + \frac{1}{6}(\varepsilon_{x}+\varepsilon_{y})+ \varepsilon_{x}\varepsilon_{y}
$$
And we notice that, every outcoming $z$ has 6 ways to get with all $x,y \in \{1,2,3,4,5,6\}$ counted twice. That gives
$$
    P[\{z\}] = \frac{1}{6} + 2 \sum_{i = 1}^{6} \varepsilon_{i} + \sum_{x,y ,\ x+y = z-1 \ \text{mod} \ 6}\varepsilon_{x}\varepsilon_{y}
$$
Note that since $\sum_{i \in \{1,2,3,4,5,6\}} P[\{i\}] = 1$,
$$
    \sum_{i = 1}^{6} \varepsilon_{i} = 0
$$
For the term
$$
\sum_{x,y ,\ x+y = z-1 \ \text{mod} \ 6}\varepsilon_{x}\varepsilon_{y}
$$
The maximum is cutted to $\frac{1}{12^2} \times 6 = \frac{1}{24} = \frac{1}{2} \times \frac{1}{12}$
So it is halved.

\section*{Exercise 1.4}
Here, we consider the probability that the family has two boys $P(BB|B_{July})$, then the probability for the family to have a girl is 
simply $1-P(BB|B_{July})$.
This probability can be calculated using Bayes's Rule
$$
    P(BB|B_{July}) = \frac{P(BB \cup B_{July})}{P(B_{July})}
$$
For the numerator, we have
$P(BB) = 1/3$ and $P(B_{July}) = 1 - (\frac{11}{12})^{2} = \frac{23}{144}$ (the probability is 1 - the probability for both boys born in other months).
For the denominator, we can decompose it as 
$$
    P(B_{July}) = P(B_{July}|BB)P(BB) + P(B_{July}|BG)P(BG) + P(B_{July}|GB)P(GB) = \frac{1}{3}\frac{1}{12} \times 2 + \frac{1}{3}\frac{23}{144}
$$
Then we can calculate the result
$$
    P(GB,BG|B_{July}) = \frac{24}{47}
$$\cite{article}

\section*{Exercise 1.5}
\subsection*{i)}
Both Alice and Bob has $\frac{1}{3}$ probability to lose due to open two wrong doors. So they will succeed with probability $(1-\frac{1}{3})^{2} = \frac{4}{9}$

\subsection*{ii)}
\begin{enumerate}
    \item Alice first open door A. If it is the combination, then open B. Else, open C.
    \item Bob open door B. If it is the goat, open C. Else, open A.
\end{enumerate}
Note that this leads to winning rate of $\frac{2}{3}$. Since if Alice win, there are only three patterns left:
\begin{enumerate}
    \item (combination, safe, goat)
    \item (goat, combination, safe)
    \item (safe, combination, goat)
    \item (safe, goat, combination)
\end{enumerate}
Then by simply open door B, Bob can rule out three conditions, including target cases, and the fourth case.
So Bob can ensure success. The total probability is thus $\frac{2}{3}$

\subsection*{iii)}
Because $\frac{2}{3}$ is the highest probability for Alice, since she has no extra information about the objects. And since we Bob ensures the success, we have no possibly higher
probability.

\section*{Exercise 1.6}
\subsection*{i)}
Use the Bayes's Rule
$$
P[d|p] = \frac{P[p | d] \cdot P[d] }{P[p |d]P[d] + P[p | h]P[h]} = \frac{95\% \cdot 1\%}{95\% \cdot 1\% + 30\% \cdot 99\%} = 0.031
$$

$$
P[d|n] = \frac{P[n | d] \cdot P[d]}{P[n |d]P[d] + P[n | h]P[h]} = \frac{5\% \cdot 1\%}{5\%\cdot 1\% + 70\%\cdot 99\%} = 7.21 \times 10^{-4}
$$

$$
P[h|p] = \frac{P[p|h]\cdot P[h]}{P[p|h]\cdot P[h] + P[p|d]\cdot P[d]} = \frac{30\% \cdot 99\%}{30\% \cdot 99\% + 95\% \cdot 1\%} = 0.969
$$

$$
P[h|n] = \frac{P[n|h]\cdot P[h]}{P[n|h] \cdot P[h] + P[n|d]\cdot P[d]} = \frac{70\% \cdot 99\%}{70\% \cdot 99\% + 5\% \cdot 1\%} = 0.993
$$
\subsection*{ii)}
Although tested negative, the people still have a large chance to be healthy. And we cannot deny the fact that the test is already accurate enough.
So if we get a positive person, we'd better test him/her again.

\subsection*{iii)}
Change all the $1\%$ to $30\%$ in i) and we get the result
$$
    P[d|p] = 0.576
$$
$$
    P[d|n] = 0.030
$$

$$
    P[h|p] = 0.424
$$

$$
    P[h|n] = 0.970
$$

\subsection*{iv)}
If one is tested to be positivem then you have a significantly higher posssibility to catch COVID-19. But one test can still not ensure you get the disease.
We need to test him/her again.

\newpage
\printbibliography

\end{document}